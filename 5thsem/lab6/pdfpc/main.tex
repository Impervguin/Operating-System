\ssr{ПРОГРАММНОЕ ОБЕСПЕЧЕНИЕ}

Особенностью реализованного сервера является то, что выполнение нескольких клиентов на одной машине приводят к ошибкам, так как ответы сервера клиентам перемешиваются. В случае же запуска клиентов на разных ip-адресах разработанное ПО работает корректно.

\begin{lstlisting}[caption={<<Файл pc.x для генерации сервера и скелетонов>>}]
	/*
	* filename: pc.x
	* function: Define constants, non-standard data types and the calling process in remote calls
	*/
	
	
	program PC_PROG
	{
		version PC_VERSION
		{
			char consume(void) = 1;
			char produce(void) = 2;
		} = 1; /* Version number = 1 */
	} = 0x20000001; /* RPC program number */
\end{lstlisting}

Для генерации файлов используется команда:

\begin{lstlisting}
	rpcgen pc.x -a -M
\end{lstlisting}

\begin{lstlisting}[caption={<<Модифицированный файл pc\_svc.c с кодом многопоточного rpc сервер>>}]
	/*
	* Please do not edit this file.
	* It was generated using rpcgen.
	*/
	
	#include "pc.h"
	#include <stdio.h>
	#include <stdlib.h>
	#include <rpc/pmap_clnt.h>
	#include <string.h>
	#include <memory.h>
	#include <sys/socket.h>
	#include <netinet/in.h>
	
	#ifndef SIG_PF
	#define SIG_PF void(*)(int)
	#endif
	
	pthread_t p_thread;
	pthread_attr_t attr;
	
	// У всех функций нет аргументов, создан для rpc
	union argument_t {
		int fill;
	};
	void*
	serv_request(void *data)
	{
		struct thr_data
		{
			struct svc_req *rqstp;
			SVCXPRT *transp;
			union argument_t argument;
		} *ptr_data;
		union argument_t argument;
		union {
			char consume_1_res;
			char produce_1_res;
		} result;
		bool_t retval;
		xdrproc_t _xdr_argument, _xdr_result;
		bool_t (*local)(char *, void *, struct svc_req *);
		ptr_data = (struct thr_data *) data;
		struct svc_req *rqstp = ptr_data->rqstp;
		register SVCXPRT *transp = ptr_data->transp;
		argument = ptr_data->argument;
		switch (rqstp->rq_proc) {
			case NULLPROC:
			(void) svc_sendreply (transp, (xdrproc_t) xdr_void, (char *)NULL);
			return NULL;
			case consume:
			_xdr_argument = (xdrproc_t) xdr_void;
			_xdr_result = (xdrproc_t) xdr_char;
			local = (bool_t (*) (char *, void *,  struct svc_req *))consume_1_svc;
			break;
			case produce:
			_xdr_argument = (xdrproc_t) xdr_void;
			_xdr_result = (xdrproc_t) xdr_char;
			local = (bool_t (*) (char *, void *,  struct svc_req *))produce_1_svc;
			break;
			default:
			svcerr_noproc (transp);
			return NULL;
		}
		retval = (bool_t) (*local)((char *)&argument, (void *)&result, rqstp);
		if (retval > 0 && !svc_sendreply(transp, (xdrproc_t) _xdr_result, (char *)&result)) {
			svcerr_systemerr (transp);
		}
		if (!svc_freeargs (transp, (xdrproc_t) _xdr_argument, (caddr_t) &argument)) {
			fprintf (stderr, "%s", "unable to free arguments");
			exit (1);
		}
		if (!pc_prog_1_freeresult (transp, _xdr_result, (caddr_t) &result))
		fprintf (stderr, "%s", "unable to free results");
		
		return NULL;
	}
	
	static void
	pc_prog_1(struct svc_req *rqstp, register SVCXPRT *transp) {
		xdrproc_t _xdr_argument;
		
		// выделение структуры, для передачи обработчику в отдельном потоке
		struct data_str
		{
			struct svc_req *rqstp;
			SVCXPRT *transp;
			union argument_t argument;
		} *data_ptr=(struct data_str*)malloc(sizeof(struct data_str));
		if (data_ptr == NULL) {
			fprintf (stderr, "%s", "unable to allocate memory.");
			exit(1);
		}
		// получение аргументов вызова
		switch (rqstp->rq_proc) {
			case NULLPROC:
			(void) svc_sendreply (transp, (xdrproc_t) xdr_void, (char *)NULL);
			return ;
			break;
			case produce:
			_xdr_argument = (xdrproc_t) xdr_void;
			break;
			case consume:
			_xdr_argument = (xdrproc_t) xdr_void;
			break;
			default:
			svcerr_noproc (transp);
			return ;
			break;
		}
		// svc_getargs не потокобезопасна, поэтому, если аргументы есть, их получение происходит на этом этапе
		memset ((char *)&data_ptr->argument, 0, sizeof (data_ptr->argument));
		if (!svc_getargs (transp, (xdrproc_t) _xdr_argument, (caddr_t) &data_ptr->argument)) {
			svcerr_decode (transp);
			return;
		}
		data_ptr->rqstp = rqstp;
		data_ptr->transp = transp;
		
		// Создание detached потока с обработчиком
		pthread_attr_init(&attr);
		pthread_attr_setdetachstate(&attr,PTHREAD_CREATE_DETACHED);
		pthread_create(&p_thread,&attr,serv_request,(void *)data_ptr);
	}
	
	int
	main (int argc, char **argv)
	{
		register SVCXPRT *transp;
		
		// Инициализация семафоров и буфера
		if (init_pc() != 0) {
			fprintf (stderr, "%s", "unable to initialize.");
			exit(1);
		}
		pmap_unset (PC_PROG, PC_VERSION);
		transp = svcudp_create(RPC_ANYSOCK);
		if (transp == NULL) {
			fprintf (stderr, "%s", "cannot create udp service.");
			exit(1);
		}
		if (!svc_register(transp, PC_PROG, PC_VERSION, pc_prog_1, IPPROTO_UDP)) {
			fprintf (stderr, "%s", "unable to register (PC_PROG, PC_VERSION, udp).");
			exit(1);
		}
		transp = svctcp_create(RPC_ANYSOCK, 0, 0);
		if (transp == NULL) {
			fprintf (stderr, "%s", "cannot create tcp service.");
			exit(1);
		}
		if (!svc_register(transp, PC_PROG, PC_VERSION, pc_prog_1, IPPROTO_TCP)) {
			fprintf (stderr, "%s", "unable to register (PC_PROG, PC_VERSION, tcp).");
			exit(1);
		}
		svc_run ();
		fprintf (stderr, "%s", "svc_run returned");
		exit (1);
		/* NOTREACHED */
	}
\end{lstlisting}

\begin{lstlisting}[caption={<<Файл скелетона pc\_server.c с реализацией функций производителей и потребителей>>}]
	/*
	* This is sample code generated by rpcgen.
	* These are only templates and you can use them
	* as a guideline for developing your own functions.
	*/
	
	#include "pc.h"
	
	#include <sys/sem.h>
	
	#define SEMAPHORE_EMPTY 0
	#define SEMAPHORE_FULL 1
	#define SEMAPHORE_BINARY 2
	
	#define p -1
	#define v 1
	
	struct sembuf start_consume[] = {{SEMAPHORE_FULL, p, 0}, {SEMAPHORE_BINARY, p, 0}};
	struct sembuf end_consume[] = {{SEMAPHORE_BINARY, v, 0}, {SEMAPHORE_EMPTY, v, 0}};
	
	struct sembuf start_produce[] = {{SEMAPHORE_EMPTY, p, 0}, {SEMAPHORE_BINARY, p, 0}};
	struct sembuf end_produce[] = {{SEMAPHORE_BINARY, v, 0}, {SEMAPHORE_FULL, v, 0}};
	
	#define BUFFER_SIZE 1000
	char *buf;
	char *current_consume;
	char *current_produce;
	char next_letter;
	int semid;
	
	int init_pc() {
		buf = malloc(sizeof(char) * BUFFER_SIZE);
		if (buf == NULL) {
			perror("malloc");
			return 1;
		}
		current_consume = buf;
		current_produce = buf;
		next_letter = 'a';
		semid = semget(IPC_PRIVATE, 3, IPC_CREAT | 0666);
		if (semid == -1) {
			perror("semget");
			return 1;
		}
		if (semctl(semid, SEMAPHORE_EMPTY, SETVAL, BUFFER_SIZE) == -1) {
			perror("semctl");
			return 1;
		}
		if (semctl(semid, SEMAPHORE_BINARY, SETVAL, 1) == -1) {
			perror("semctl");
			return 1;
		}
		if (semctl(semid, SEMAPHORE_FULL, SETVAL, 0) == -1) {
			perror("semctl");
			return 1;
		}
		return 0;
	}
	
	bool_t
	consume_1_svc(void *argp, char *result, struct svc_req *rqstp)
	{
		bool_t retval;
		int err = semop(semid, start_consume, 2);
		if (err == -1) {
			perror("semop\n");
			exit(1);
		}
		*result = *current_consume;
		current_consume++;
		if (current_consume - buf == BUFFER_SIZE) {
			current_consume = buf;
		}
		err = semop(semid, end_consume, 2);
		if (err == -1) {
			perror("semop\n");
			exit(1);
		}
		return TRUE;
	}
	
	bool_t
	produce_1_svc(void *argp, char *result, struct svc_req *rqstp)
	{
		bool_t retval;
		int err = semop(semid, start_produce, 2);
		if (err == -1) {
			perror("semop\n");
			exit(1);
		}
		*current_produce = next_letter;
		*result = next_letter;
		next_letter++;
		if (next_letter > 'z') {
			next_letter = 'a';
		}
		current_produce++;
		if (current_produce - buf == BUFFER_SIZE) {
			current_produce = buf;
		}
		err = semop(semid, end_produce, 2);
		if (err == -1) {
			perror("semop\n");
			exit(1);
		}
		return TRUE;
	}
	
	int
	pc_prog_1_freeresult (SVCXPRT *transp, xdrproc_t xdr_result, caddr_t result)
	{
		xdr_free (xdr_result, result);
		/*
		* Insert additional freeing code here, if needed
		*/
		return 1;
	}
	
\end{lstlisting}
	
\begin{lstlisting}[caption={<<Файл скелетона pc\_client.c с клиентам потребителями и производителями>>}]
	/*
	* This is sample code generated by rpcgen.
	* These are only templates and you can use them
	* as a guideline for developing your own functions.
	*/
	
	#include "pc.h"
	#include <string.h>
	#include <unistd.h>
	#include <stdlib.h>
	
	void
	pc_prog_consume(char *host)
	{
		CLIENT *clnt;
		enum clnt_stat retval_1;
		char result_1;
		char *produce_1_arg;
		#ifndef	DEBUG
		clnt = clnt_create (host, PC_PROG, PC_VERSION, "tcp");
		if (clnt == NULL) {
			clnt_pcreateerror (host);
			exit (1);
		}
		#endif	/* DEBUG */
		srand(getpid());
		while (1) {
			sleep(rand() % 3 + 1);	
			retval_1 = consume_1((void*)&produce_1_arg, &result_1, clnt);
			if (retval_1 != RPC_SUCCESS) {
				clnt_perror (clnt, "call failed");
			}
			printf("get %c\n", result_1);
		}
		#ifndef	DEBUG
		clnt_destroy (clnt);
		#endif	 /* DEBUG */
	}
	
	void
	pc_prog_produce(char *host)
	{
		CLIENT *clnt;
		enum clnt_stat retval_1;
		char result_1;
		char *produce_1_arg;
		#ifndef	DEBUG
		clnt = clnt_create (host, PC_PROG, PC_VERSION, "tcp");
		if (clnt == NULL) {
			clnt_pcreateerror (host);
			exit (1);
		}
		#endif	/* DEBUG */
		srand(getpid());
		while (1) {
			sleep(rand() % 3 + 1);
			
			retval_1 = produce_1((void*)&produce_1_arg, &result_1, clnt);
			if (retval_1 != RPC_SUCCESS) {
				clnt_perror (clnt, "call failed");
			}
			printf("put %c\n", result_1);
		}
		#ifndef	DEBUG
		clnt_destroy (clnt);
		#endif	 /* DEBUG */
	}
	
	
	int
	main (int argc, char *argv[])
	{
		char *host;
		
		if (argc < 3) {
			printf ("usage: %s server_host pc_type\n", argv[0]);
			printf ("pc_type = 1 - consumer\n");
			printf ("pc_type = 2 - producer\n");
			exit (1);
		}
		if (strcmp(argv[2], "1") != 0 && strcmp(argv[2], "2") != 0) {
			printf ("usage: %s server_host pc_type\n", argv[0]);
			printf ("pc_type = 1 - consumer\n");
			printf ("pc_type = 2 - producer\n");
			exit (1);
		}
		host = argv[1];
		if (strcmp(argv[2], "1") == 0) {
			pc_prog_consume (host);
		}
		else {
			pc_prog_produce (host);
		}
		exit (0);
	}
\end{lstlisting}

Файл <<pc\_clnt.c>> с кодом клиента не изменяется, а в файл <<pc.h>> необходимо добавить строчку:

\begin{lstlisting}
	int init_pc();
\end{lstlisting}

В <<Makefile.pc>> нужно изменить две переменные:
\begin{lstlisting}
	CFLAGS = -g -I/usr/include/tirpc
	LDLIBS = -lnsl -lpthread -ltirpc
\end{lstlisting}

