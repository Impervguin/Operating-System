\documentclass[a4paper,14pt]{article}
\usepackage{graphicx}
\usepackage[utf8]{inputenc}
\usepackage[russian]{babel}
\usepackage{geometry}
\usepackage{listings}
\usepackage{xcolor}

\lstdefinestyle{asm}{
	language={[x86masm]Assembler},
	backgroundcolor=\color{white},
	basicstyle=\ttfamily,
	keywordstyle=\color{blue},
	stringstyle=\color{red},
	commentstyle=\color{gray},
	numbers=left,
	numberstyle=\tiny,
	stepnumber=1,
	numbersep=5pt,
	frame=single,
	tabsize=4,
	captionpos=b,
	breaklines=true
}

\lstset{
	literate=
	{а}{{\selectfont\char224}}1
	{б}{{\selectfont\char225}}1
	{в}{{\selectfont\char226}}1
	{г}{{\selectfont\char227}}1
	{д}{{\selectfont\char228}}1
	{е}{{\selectfont\char229}}1
	{ё}{{\"e}}1
	{ж}{{\selectfont\char230}}1
	{з}{{\selectfont\char231}}1
	{и}{{\selectfont\char232}}1
	{й}{{\selectfont\char233}}1
	{к}{{\selectfont\char234}}1
	{л}{{\selectfont\char235}}1
	{м}{{\selectfont\char236}}1
	{н}{{\selectfont\char237}}1
	{о}{{\selectfont\char238}}1
	{п}{{\selectfont\char239}}1
	{р}{{\selectfont\char240}}1
	{с}{{\selectfont\char241}}1
	{т}{{\selectfont\char242}}1
	{у}{{\selectfont\char243}}1
	{ф}{{\selectfont\char244}}1
	{х}{{\selectfont\char245}}1
	{ц}{{\selectfont\char246}}1
	{ч}{{\selectfont\char247}}1
	{ш}{{\selectfont\char248}}1
	{щ}{{\selectfont\char249}}1
	{ъ}{{\selectfont\char250}}1
	{ы}{{\selectfont\char251}}1
	{ь}{{\selectfont\char252}}1
	{э}{{\selectfont\char253}}1
	{ю}{{\selectfont\char254}}1
	{я}{{\selectfont\char255}}1
	{А}{{\selectfont\char192}}1
	{Б}{{\selectfont\char193}}1
	{В}{{\selectfont\char194}}1
	{Г}{{\selectfont\char195}}1
	{Д}{{\selectfont\char196}}1
	{Е}{{\selectfont\char197}}1
	{Ё}{{\"E}}1
	{Ж}{{\selectfont\char198}}1
	{З}{{\selectfont\char199}}1
	{И}{{\selectfont\char200}}1
	{Й}{{\selectfont\char201}}1
	{К}{{\selectfont\char202}}1
	{Л}{{\selectfont\char203}}1
	{М}{{\selectfont\char204}}1
	{Н}{{\selectfont\char205}}1
	{О}{{\selectfont\char206}}1
	{П}{{\selectfont\char207}}1
	{Р}{{\selectfont\char208}}1
	{С}{{\selectfont\char209}}1
	{Т}{{\selectfont\char210}}1
	{У}{{\selectfont\char211}}1
	{Ф}{{\selectfont\char212}}1
	{Х}{{\selectfont\char213}}1
	{Ц}{{\selectfont\char214}}1
	{Ч}{{\selectfont\char215}}1
	{Ш}{{\selectfont\char216}}1
	{Щ}{{\selectfont\char217}}1
	{Ъ}{{\selectfont\char218}}1
	{Ы}{{\selectfont\char219}}1
	{Ь}{{\selectfont\char220}}1
	{Э}{{\selectfont\char221}}1
	{Ю}{{\selectfont\char222}}1
	{Я}{{\selectfont\char223}}1
}


\geometry{pdftex, left = 3cm, right = 10mm, top = 2cm, bottom = 2cm}

\begin{document}
	
	\begin{titlepage}
		\noindent
		\begin{tabular}{|c|c|}	
			\hline
			\noindent	
			\begin{minipage}{0.14\textwidth}
				\includegraphics[width=\linewidth]{img/b_logo}
			\end{minipage} &
			\noindent
			\begin{minipage}{0.75\textwidth}\centering
				\textbf{\newline Министерство науки и высшего образования Российской Федерации}\\
				\textbf{Федеральное государственное бюджетное образовательное учреждение высшего образования}\\
				\textbf{«Московский государственный технический университет имени Н.Э.~Баумана}\\
				\textbf{(национальный исследовательский университет)»}\\
				\textbf{(МГТУ им. Н.Э.~Баумана)}
			\end{minipage} \\
			\hline	
		\end{tabular}
		\newline\newline
		\noindent ФАКУЛЬТЕТ \underline{«Информатика и системы управления»} \newline\newline
		\noindent КАФЕДРА \underline{«Программное обеспечение ЭВМ и информационные технологии»}\newline\newline\newline\newline\newline\newline\newline
		
		\begin{center}
			\noindent\begin{minipage}{1.0\textwidth}\centering
				\Large\textbf{   ~~~ Лабораторная работа №1}\newline
				\textbf{по дисциплине "Операционные системы"}\newline
				\textbf{"Дизассемблирование INT 8h"}\newline
			\end{minipage}
		\end{center}
		\noindent\textbf{Студент:} \underline{Шахнович Д.С.}\newline\newline
		\noindent\textbf{Группа:} \underline{ИУ7-52Б}\newline\newline
		\noindent\textbf{Преподаватель:} \underline{Рязанова Н.Ю.}\newline
		
		\begin{center}
			\mbox{}
			\vfill
			Москва, \the\year ~г.
		\end{center}
		\clearpage
	\end{titlepage}
	
	\section{Листинги дизассемблированного кода}
	\subsection{Листинг int 8h}
	
	\begin{lstlisting}[style={asm}]
      Temp.lst				Sourcer	v5.10   12-Sep-24   2:18 pm   Page 1

; Вызов сабрутины
020A:0746  E8 0070				call	sub_1			; (07B9)

; Сохранение регистров
020A:0749  06					push	es
020A:074A  1E					push	ds
020A:074B  50					push	ax
020A:074C  52					push	dx

; Установка ds - сегмент bios, es - сегмент векторов прерывания
020A:074D  B8 0040				mov	ax,40h
020A:0750  8E D8				mov	ds,ax
020A:0752  33 C0				xor	ax,ax			; Zero register
020A:0754  8E C0				mov	es,ax

; Инкремент младшего слова счётчика тиков
020A:0756  FF 06 006C				inc	word ptr ds:[6Ch]	; (0040:006C=4DD4h)
020A:075A  75 04				jnz	loc_1			; Jump if not zero

; Если младшее слово обнулилось, то инкремент старшего слова счётчика тиков
020A:075C  FF 06 006E				inc	word ptr ds:[6Eh]	; (0040:006E=0Eh)
020A:0760			loc_1:

Если старшее слово равно 18h(24) и младшее равно B0h(176) то это значит, что прошёл день
020A:0760  83 3E 006E 18			cmp	word ptr ds:[6Eh],18h	; (0040:006E=0Eh)
020A:0765  75 15				jne	loc_2			; Jump if not equal
020A:0767  81 3E 006C 00B0			cmp	word ptr ds:[6Ch],0B0h	; (0040:006C=4DD4h)
020A:076D  75 0D				jne	loc_2			; Jump if not equal
; Зануляем младшее и старшее слова счётчика тиков
020A:076F  A3 006E				mov	word ptr ds:[6Eh],ax	; (0040:006E=0Eh)
020A:0772  A3 006C				mov	word ptr ds:[6Ch],ax	; (0040:006C=4DD4h)

; Записываем значение 1 по 0040h:0070h
020A:0775  C6 06 0070 01			mov	byte ptr ds:[70h],1	; (0040:0070=0)

; Устанавливаем 3-й бит в al
020A:077A  0C 08				or	al,8
020A:077C			loc_2:

; Работа с моторчиком дисковода
020A:077C  50					push	ax
; Декремент счётчика времени до отключения моторчика
020A:077D  FE 0E 0040				dec	byte ptr ds:[40h]	; (0040:0040=0E2h)
; Если счётчик обнулился, то
020A:0781  75 0B				jnz	loc_3			; Jump if not zero
; Сбрасываем флаг работы моторчика
020A:0783  80 26 003F F0			and	byte ptr ds:[3Fh],0F0h	; (0040:003F=0)
; Отправляем команду отключения моторчика дисквода 0Ch на порт 3F2h
020A:0788  B0 0C				mov	al,0Ch
020A:078A  BA 03F2				mov	dx,3F2h
020A:078D  EE					out	dx,al			; port 3F2h, dsk0 contrl output
020A:078E			loc_3:
020A:078E  58					pop	ax

; Проверка флага чётности(PF - 2-й бит)
020A:078F  F7 06 0314 0004			test	word ptr ds:[314h],4	; (0040:0314=3200h)
020A:0795  75 0C				jnz	loc_4			; Jump if not zero
; Загрузка младшего байта регистра флагов а ah
020A:0797  9F					lahf				; Load ah from flags
; Смена ah, al
020A:0798  86 E0				xchg	ah,al
; Сохраняем регистр флагов 
020A:079A  50					push	ax
; Косвенный вызов прерывания 1Ch
020A:079B  26: FF 1E 0070			call	dword ptr es:[70h]	; (0000:0070=6ADh)
020A:07A0  EB 03				jmp	short loc_5		; (07A5)
020A:07A2  90					nop
020A:07A3			loc_4:
; Прямой вызов 1Ch, если прерывания не запрещены
020A:07A3  CD 1C				int	1Ch			; Timer break (call each 18.2ms)
020A:07A5			loc_5:
; Вызов сабрутины
020A:07A5  E8 0011				call	sub_1			; (07B9)
; Завершение прерывания(сброс контроллера прерываний), разрешения прерывания с более низкими приоритетами
020A:07A8  B0 20				mov	al,20h			; ' '
020A:07AA  E6 20				out	20h,al			; port 20h, 8259-1 int command
;  al = 20h, end of interrupt
; Восстановление значений регистров 
020A:07AC  5A					pop	dx
020A:07AD  58					pop	ax
020A:07AE  1F					pop	ds
020A:07AF  07					pop	es

020A:07B0  E9 FE99				jmp	$-164h  ; 07B0h - 164h = 064Ch
; ...
020A:064C  1E					push	ds
020A:064D  50					push	ax
; ...
020A:06AA  58					pop	ax
020A:06AB  1F					pop	ds
020A:06AC  CF					iret ; возврат из прерывания
	\end{lstlisting}
	\pagebreak
	\subsection{Листинг процедуры sub\_1}
	\begin{lstlisting}[style={asm}]
				sub_1		proc	near
; Сохранение значений регистров
020A:07B9  1E					push	ds
020A:07BA  50					push	ax
; Установка ds - сегмент bios
020A:07BB  B8 0040				mov	ax,40h
020A:07BE  8E D8				mov	ds,ax

; Загрузка в ah младшего бита регистра флагов
020A:07C0  9F					lahf				; Load ah from flags
; Проверка DF и старшего бита IOPL
020A:07C1  F7 06 0314 2400			test	word ptr ds:[314h],2400h	; (0040:0314=3200h)
020A:07C7  75 0C				jnz	loc_7			; Jump if not zero
; Сброс флага прерываний IF в BIOS
020A:07C9  F0> 81 26 0314 FDFF	                           lock	and	word ptr ds:[314h],0FDFFh	; (0040:0314=3200h)
020A:07D0			loc_6:
; Загрузка из ah младшего бит флагов
020A:07D0  9E					sahf				; Store ah into flags
; Восстановление регистров
020A:07D1  58					pop	ax
020A:07D2  1F					pop	ds
020A:07D3  EB 03				jmp	short loc_8		; (07D8)
020A:07D5			loc_7:
; Сброс флага прерываний IF
020A:07D5  FA					cli				; Disable interrupts
020A:07D6  EB F8				jmp	short loc_6		; (07D0)
020A:07D8			loc_8:
; Возврат из сабрутины
020A:07D8  C3					retn
sub_1		endp
		
	\end{lstlisting}
	\pagebreak
	\section{Схемы алгоритмов}
	\subsection{Cхема алгоритма int 8h}
	\begin{center}
		\includegraphics[height=0.5\textheight]{img/main1}

	\end{center}
	\pagebreak
	\begin{center}
	\includegraphics[height=0.85\textheight]{img/main2}
	\end{center}
	\pagebreak
	\begin{center}
	\includegraphics[height=0.5\textheight]{img/main3}
	\end{center}
	\pagebreak
	\begin{center}
	\includegraphics[height=0.6\textheight]{img/main4}
	\end{center}
	\pagebreak
	\subsection{Cхема алгоритма sub\_1}
	\begin{center}
		\includegraphics[height=0.67\textheight]{img/sub}
	\end{center}
	\pagebreak
	\section{Заключение}
	В ходе работы мною был разобран и изучен дизассемблированный код обработчика int 8h, выявлены его основные функции:
	\begin{enumerate}
		\item Инкремент счётчика тиков
		\item Контроль счётчика тиков при наступлении нового дня
		\item Декремент счётчика тиков до выключения моторчика дисковода
		\item Выключение моторчика, в случае зануления таймера
		\item Вызов обработчика прерывания int 1Ch
	\end{enumerate}
	
	
	
	
\end{document}